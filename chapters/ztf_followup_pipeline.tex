\setchapterimage[7.5cm]{palomar}
\setchapterpreamble[u]{\margintoc}
\chapter{Transient Follow-up with ZTF}
\labch{ztf_too}
\begin{fquote}[Oscar Wilde][Lady Windermere's Fan][18??]We are all in the gutter, but some of us are looking at the stars
\end{fquote}

The motivation for all real-time analysis and prompt follow-up observations of external triggers is to obtain contemporaneous data. Put another way, these observations are designed to quickly identify time-dependent transient or variable activity which would be missed by untargeted survey operations.  In the case of GW and short GRB follow-up, searches are targeted towards identifying kilonovae (KNe), of which GW170817/AT2017gfo was a spectacular well-documented first example. These KNe are inherently fast-evolving transients which will by definition not be present before the extrenal trigger. For neutrinos, there are a proliferation of possible optical transients (e.g GRBs, TDEs, SNe) and variable objects (primarily blazars) which could be identified in an optical follow-up observations. In both cases, a handful of potentially interesting extragalactic objects will need to be selected against a vast background of spurious detections, and unrelated galactic or solar-system objects.

\section{The AMPEL Follow-up Pipeline}
Given the similar challenge, a single flexible analysis, the \ztf, was developed by the author to identify candidate counterparts to GW, GRB and neutrino trigger events. ZTF is particularly well-suited to this type of extragalactic transient discovery,  because ZTF routinely images the visible Northern Sky once every three nights to a median depth of $20.5^{m}$ as part of a public survey \sidecite{2019PASP..131g8001G, 2019PASP..131a8002B}.  This provides an extensive base to distinguish between new transients that could be coincident with an external trigger, and those with previous detections. The large volumetric survey speed of ZTF often provides significant serendipitous coverage of events. This wide-field cadence can however be supplemented by dedicated Target-of-Opportunity (ToO) observations scheduled for a particular trigger through the GROWTH ToO Marshal \sidecite{2019PASP..131d8001C}. 

Following both survey and ToO observations, the raw images are processed by IPAC, and alert packets are promptly generated for significant differences relative to survey reference images \sidecite{2019PASP..131a8003M,2019PASP..131a8001P}. These alerts are then distributed as part of a near-realtime stream to brokers, and are archived in a database of alert packets at DESY as part of the local infrastructure for multi-messenger data analysis (the \emph{archive database}).

External triggers, which come in the form of probability skymaps or GCN circulars, are downloaded and parsed with \ztf. The archive database is then queried for coincident alerts, using either temporal indexing or spatial indexing depending on the size of the target region. Once loaded, these alerts are then filtered by the \ztf~using \emph{AMPEL} \sidecite{2019A&A...631A.147N}, a platform for realtime analysis of multi-messenger astronomy data. Our selection is based on an algorithm for identifying extragalactic transients \cite{2019A&A...631A.147N}. In order to identify candidate counterparts, we apply the following cuts to ToO and survey data:

\begin{itemize}
	\item We reject likely subtraction artifacts using machine learning classification and morphology cuts \sidecite{2019PASP..131c8002M}. Subtractions must be positive, i.e a candidate must be brighter than the same position in stacked reference images.
	\item We reject solar system objects through matches to known catalogues \cite{2019PASP..131a8003M}. We further remove uncatalogued solar system objects by requiring multiple detections for each candidate at the same location, separated temporally by at least 15 minutes, thereby rejecting moving objects.
	\item We reject galactic stellar sources by removing detections cross-matched to objects with measured parallax in the second data release of the \emph{GAIA} satellite \sidecite{2018A&A...616A...1G}. Objects are rejected if they have non-zero parallax with a significance of at least 3$\sigma$. We further reject likely stars with machine learning classifications, based on sources detected by Pan-STARRS1 \sidecite{2018PASP..130l8001T}, removing those objects with an estimated stellar probability greater than 80\%. 
	\item We reject AGN variability by cross-matching objects to the WISE survey. We identify likely AGN by applying a series of cuts based on measured IR colour \sidecite{2010AJ....140.1868W}.
	\item We rejected unassociated objects by requiring both spatial and temporal coincidence with a given trigger.
\end{itemize}

\section{Neutrino Follow-up}
ZTF has since its inception had a dedicated program to identify sources of high-energy astrophysical neutrinos, through targeted follow-up observations. Given the proliferation of proposed source classes, neutrino follow-up are characterised by searching relatively well-localised regions for poorly-defined objects. This is particularly true for optical follow-up, where potential counterparts include Tidal Disruption Events (TDEs), Type IIn or choked jet SNe, kilonovae, GRB afterglows, superluminous supernovae (SLSNe) or blazar flares. While pre-neutrino detections could rule out neutrino production from choked jet supernovae, in most cases we do not even have stringent constraints in when neutrinos may be expected 

 We have followed up eight neutrinos in the period from survey start on 2018 March 20 to 2020 March 31, out of a total of 31 neutrino alerts published by IceCube. Table \ref{tab:nu_alerts} summarises each neutrino alert that has been observed by ZTF. From 2019 June 17, IceCube published neutrino alerts with improved selection criteria to provide an elevated alert rate \sidecite{2019ICRC...36.1021B} (see Chapter \ref{ch:realtime} for more details). In addition to 1 of the 12 alerts under the old V1 selection, ZTF followed up 7 of the 19 alerts published under the V2 selection. In general, we aim to follow all well-localised neutrinos of likely astrophysical origin reported by IceCube which are visible to ZTF and can be observed promptly. Those alerts not observed by ZTF are summarised in Table \ref{tab:nu_non_observed}. Of those 23 alerts not followed up by ZTF, the primary reasons were proximity to the Sun (8/23), alerts with poor localisation and low astrophysical probability (6/23) and alert retraction (4/23). For events which were reported with estimates of astrophysical probability\footnote{This value was not reported for high-energy starting events (HESE) under the old IceCube alert selection, nor for one recent alert, IC200107A, that was identified outside of the standard alert criteria \sidecite{stein:gcn26655}.}, we chose not to follow up those that had both low astrophysical probability ($<$ 50\%) and large localisation regions ($>$ 10 sq.\,deg.).

\begin{table*}
	\centering
	\begin{tabular}{||c c c c c c c ||} 
		\hline
		\textbf{Event} & \textbf{R.A. (J2000)} & \textbf{Dec (J2000)} & \textbf{90\% area} & \textbf{ZTF obs} &~ \textbf{Signalness}& \textbf{Ref}\\
		& \textbf{(deg)}&\textbf{(deg)}& \textbf{(sq. deg.)}& \textbf{(sq. deg.)} &&\\
		\hline
		IC190503A & 120.28 & +6.35 & 1.94& 1.37 & 36\%&\cite{blaufuss:gcn24378,2019ATel12730....1S}\\
		IC190619A & 343.26 & +10.73 & 27.16& 21.57 & 55\%&\cite{blaufuss:gcn24910, 2019ATel12879....1S}\\
		IC190730A & 225.79 & +10.47 & 5.41& 4.52 & 67\%&\cite{stein:gcn25225, 2019ATel12967....1T, 2019ATel12974....1S}\\
		IC190922B & 5.76 & -1.57 & 4.48 & 4.09 & 51\%&\cite{blaufuss:gcn25806,2019ATel13125....1S, stein:gcn25824}\\
		\textbf{IC191001A} & \textbf{314.08} & \textbf{+12.94} & \textbf{25.53} & \textbf{20.56} & \textbf{59\%}& \textbf{\cite{stein:gcn25913,2019ATel13160....1S, stein:gcn25929}}\\
		IC200107A & 148.18 & +35.46 & 7.62 & 6.22 & - &\cite{stein:gcn26655,stein:gcn26667}\\
		IC200109A & 164.49 & +11.87 & 22.52 & 20.06 & 77\%&\cite{stein:gcn26696,reusch:gcn26747}\\
		IC200117A & 116.24 & +29.14 & 2.86 &  2.66 & 38\%&\cite{lagunas:gcn26802,reusch:gcn26813, reusch:gcn26816}\\
		\hline
	\end{tabular}
	\caption{Summary of the eight neutrino alerts followed up by ZTF, with IC191001A highlighted in bold. The 90\% area column indicates the region of sky observed at least twice by ZTF, within the reported 90\% localisation, and accounting for chip gaps. The \textit{signalness} describes the probability that each neutrino is of astrophysical origin, rather than arising from atmospheric backgrounds. One alert, IC200107A, was reported without a signalness estimate.}
	\label{tab:nu_alerts}
\end{table*}

\begin{table}
	\centering
	\begin{tabular}{||c c||} 
		\hline
		\textbf{Cause} & \textbf{Events} \\
		\hline
		\textbf{Alert Retraction} & IC180423A\cite{IC180423A}, IC181031A\cite{IC181031A},\\ &IC190205A\cite{IC190205A}, IC190529A\cite{IC190529A}\\
		&\\
		\hline
		\textbf{Proximity to Sun} & IC180908A\cite{IC180908A}, IC181014A\cite{IC181014A},\\ &IC190124A\cite{IC190124A}, IC190704A\cite{IC190704A},\\
		& IC190712A\cite{IC190712A}, IC190819A\cite{IC190819A},\\
		& IC191119A\cite{IC191119A}, IC200227A\cite{IC200227A}\\
		&\\
		\textbf{Low Altitude} & IC191215A\cite{IC191215A}\\
		&\\
		\textbf{Southern Sky} & IC190331A\cite{IC190331A}, IC190504A\cite{IC190504A}\\
		&\\
		\hline
		\textbf{Poor Signalness \& Localisation} &
		IC190221A\cite{IC190221A}, IC190629A\cite{IC190629A}, \\
		&IC190922A\cite{IC190922A}, IC191122A\cite{IC191122A}, \\
		&IC191204A\cite{IC191204A}, IC191231A\cite{IC191231A}\\
		&\\
		\hline
		\textbf{Bad Weather} & IC200120A\cite{IC200120A,IC200120A_2}\\
		&\\
		\textbf{Telescope Maintenance} & IC181023A\cite{IC181023A}\\
		&\\
		\hline
	\end{tabular}
	\caption{Summary of the 23 neutrino alerts that were not followed up by ZTF since survey start on 2018 March 20. Of these, 4/23 were retracted, 11/23 were inaccessible to ZTF for various reasons, 6/23 were deemed alerts of poor quality, while just 2/23 were alerts that were missed although they passed our criteria.}
	\label{tab:nu_non_observed}
\end{table}

Each neutrino localisation region can typically be covered by one or two ZTF observation fields. Multiple observations are scheduled for each field, with both $g$ and $r$ filters, and a separation of at least 15 minutes between images. These observations typically last for 300~s, with a typical limiting magnitude of $21.0^{m}$.  ToO observations are typically conducted on the first two nights following a neutrino alert, before swapping to serendipitous coverage as part of the public survey. We search ZTF data both preceding and following the arrival of the neutrino. Our coincidence criteria require that an object lie within the 90\% localisation region reported by IceCube in a GCN circular, and must have been detected at least once following the neutrino arrival time. These cuts typically yield $\sim$0.2 candidates per square degree of sky. Promising candidates are prioritised for spectroscopic classification, to confirm or rule out a possible association with a given neutrino. 

A selection of highlighted  results are given below. A further candidate, AT2019dsg, is described in Chapter \ref{ch:bran}.

\subsection{PKS 1502+106}
As introduced in Chapter \ref{ch:realtime}, the neutrino IC190730A was reported in spatial coincidence with PKS 1502+106, a particularly bright Flat Spectrum Radio Quasar (FSRQ) \sidecite{2019ATel12967....1T}. The object was observed by ZTF as part of ToO observations, and the object DID was also identified by our follow-up pipeline as a candidate counterpart.  The blazar was repeatedly detected as part of the routine survey operations under the name \emph{ZTF18aaqnqzx}, with both positive and negative flux changes relative to survey reference images. The blazar lightcurve is shown in Figure \ref{fig:pks1502_ligtcurve}, with the neutrino arrival time marked in blue. Consistent with observations from other observatories across many wavelengths (Chapter \ref{ch:realtime}), there was no significant flaring observed for this source coincident with the neutrino. The blazar at this point was dimmer than survey reference images, with the neutrino arriving during a year-long fading. 

\begin{figure}[!ht]
	\centering \includegraphics{ZTF/pks1502_lightcurve}
	\caption{ZTF lightcurve of blazar PKS 1502+106, with observations in g, r and i bands. The arrival time of the neutrino on 2019 July 30 is marked with a dotted line. The horizontal dashed lines indicate the magnitude of the source in reference images derived from the PanSTARRS survey.}
	\label{fig:pks1502_ligtcurve}
\end{figure}

\subsection{SN2019pqh}
As introduced in Chapter \ref{ch:realtime}, follow-up of IC190922B by ZTF identified a candidate supernova ZTF19abxtupj/SN2019pqh \sidecite{2019ATel13125....1S}. As outlined in chapter \ref{ch:sources}, neutrino emission has been predicted for core-collapse supernovae from either choked jets or interacting supernovae. SN2019pqh was not consistent with the former scenario, because it was detected prior to the neutrino. However, it was close to peak, as shown in the lightcurve in Figure \ref{fig:sn2019pqh_ligtcurve}. This was the first young supernova found in coincidence with a high-energy neutrino, which is interesting because CSM neutrino emission . 

\begin{figure}[!ht]
	\centering \includegraphics{ZTF/sn2019pqh_lightcurve}
	\caption{ZTF lightcurve of SN2019pqh, with observations in g and r band. The arrival time of the neutrino on 2019 September 22 is marked with a dotted line, and the supernova is detected in the subsequent ToO obsersations.}
	\label{fig:sn2019pqh_ligtcurve}
\end{figure}

This object had first been detected on A subsequent classification of the object as a Type II supernova was reported by ePESSTO (cite). 

A high-resolution spectrum of the object was obtained by De Sharma on 2019 September 28, using the (LRIS) spectrograph at the Keck 1 observatory (see Figure \ref{fig:sn2019pqh_spectrum}). The host galaxy, clearly visible in Figure N, had a spectroscopic redshift of z=0.134 measured by the Sloan .... (SDSS). Peaks are clearly visible in the spectrum matching hydrogen lines for this redshift, confirming that the transient is indeed coincident with the apparent host galaxy.

\begin{figure}[!ht]
	\centering \includegraphics{ZTF/sn2019pqh_spectrum}
	\caption{Spectrum of SN2019pqh, taken on x. The prominent Balmer lines are highlighted in orange, from which a redshift of 0.134 is derived. A template-matching classification using SNID yields a match to a Type IIb supernova (SN1993J) 2 days before peak cite. }
	\label{fig:sn2019pqh_spectrum}
\end{figure}

\section{Gravitational Waves and Gamma-ray Bursts}
In contrast to neutrinos, follow-up of GW and GRB triggers is more akin to searching for a needle in a haystack. The localisations tend to be much larger than for neutrinos, but with the well-defined aim of finding kilonovae. In this case, we at least know what our needle looks like. GW170817 provided an observational confirmation of theoretical predictions of kilonovae, namely that they are faint, fast-evolving, red transients which are generated following mergers of neutron stars. Though a trigger event may cover several hundred or thousand square degrees, these constraints enable us to efficiently reject the vast majority of candidates. Additional temporal coincidence is ensured by requiring that candidates are not detected prior to merger/burst, and spatial coincidence can be ensured using trigger probability skymaps integrated out to predefined confidence intervals (typically 90\% or 95\%). Unique among triggers, GW events are also typically reported with estimated distance ranges derived from template-matching. For candidates which have a resolved host galaxy, we can determine their distance using spectroscopic surveys such as SDSS, as well as photometric redshift estimates, and veto those candidates lying outside the desired distance range.

The third observing run of the LIGO/VIRGO (O3) extended from X to y, and for this period ZTF has followed-up LVc GW triggers. In this period, there were BNs candidates and Y BBH candidates.

There have been some predictions of EM signatures for BBH events...
