\documentclass[]{article}
\begin{document}
\section{Introduction}
\section{Neutrino Theory}
\section{The IceCube Neutrino Observatory}
\section{The Zwicky Transient Facility (ZTF)}
\section{Sources of Astrophysical Neutrinos}
As far back as X, when nuclear fusion was proposed as a mechanism to power the sun, it was expected that a flux of solar neutrinos should be detectable on Earth. This prediction was confirmed by early neutrino detectors, ZZ and Z, which measured the flux xyc.
The discovery of a source of neutrinos from beyond the solar system followed soon after, with the detection of nearby supernova SN1987A. Nearby supernova occur either within our galaxy, or in the neighbouring Magellanic clouds, at a rate of a couple per century. In thie case of SN1987A, the detection of the supernova was preceded by simultaneous detections of an elevated neutrino flux in multiple detectors on Earth. The coincident detection of photons and neutrinos marjked the first multi-messenger detection of an astrophysical source.

It is now well-understood that there is a diffuse flux of MeV-neutrinos produced from SN?, and preparations are underway for the inevitable next nearby SN, coordinated by the SuperNova Early Warning System (SNEWS). Given the increased volume of current-generation neutrino detectors, the next nearby supernova will be measured with much greater precision. IceCube itself will contribute to this, through a dedicated supernova detection system. Given the detector geometry, IceCube will not be able to identify individual neutrino events, nor reconstruct their directions. However, an increase in DOM noise will be clearly measurable, and likely with sufficiently resolution to resolve the time-evolution of the signal. 

In both cases, the neutrinos detected at MeV-energies.  However, in light of the discovery of a flux of high-energy astrophysical neutrinos by IceCube, a new branch of astronomy has developed searching for sources of these astrophysical neutrinos. The central fact for neutrino astronomy, at least in the ~TeV-PeV range for which IceCube is sensitive, is that at lower energies there is an overwhelming background, while at higher energies, statistics are poor. This is illustrated in Fig. The power of IceCube to detect an astrophysical flux depends on the degree to which this differs from the atmospheric background. Above all, the expected energy distribution for most astrophysical sources likely differs from the soft-spectrum atmospheric neutrino flux. In addition, the spatial distribution of the background is broadly isotropic, and uniform as a function of right ascension. The temporal distribution of this background, beyond small-scale season variations, is also reasonably isotropic. Foreground fuxes which differ substantially will be most easily detected.

The expected degree of anisotropy in the extragalactic astrophysical neutrino flux strongly depends on the properties of the assumed sources, in particular their density and source evolution. The source evolution describes the rate or density evolution of astrophysical objects as a function of redshift. For sources with a negative source evolution, the density in the local universe is greater than at high redshift, so there are fewer neutrinos produced by distant, unresolved sources. Conversely, for a strongly positive source evolution, we expect a greater fraction of neutrinos to arrive from unresolved sources. When local density and source evolution are coupled, we then find arrive at a flux which anisotropic to varying degrees. Given that the atmospheric backgrounds are broadly isotropic, and uniform as a function of right ascension, the sensitivity of IceCube to a given astrophysical flux will depend strongly on whether that flux is sufficiently spatially anistotropic to distinguish it from atmospheric backgrounds. A higher-density source population, with positive evolution, would be much harder to identify that a low-density one with negative evolution. Anisotropy in neutrino arrival times can be an additional metric to distinguish astrophysical neutrinos. For sources which are variable or transient, neutrino emission is only expected for distinct time periods, eliminating background events.

In general, given knowledge about the background, the most agnostic methods to identify neutrino sources look for clustering within the data without reference to external datasets. Such auto-correlation analyses can be done with either a time-integrated or time-dependent all-sky likelihood scan. The result of such an analysis is a pixelised likelihood map. Comparisons of this map can then be made to expectations from background, either using the single "hottest spot" in the sky, or by comparing the distribution of hotspots. No significant excess was found for either approach, providing significant general constraints on neutrino source populations, . Given the current detector volumes and resolution, as well as the lack of observed lower-significance overfluctuations, it appears that the astrophysical neutrino flux is not sufficiently anisotropic for auto-correlation analyses to discover a neutrino source in the near future.

As an alternative to agnostic searches, specific source hypothesis tests can be substantially more sensitive. Given the position of a known astrophysical object, the threshold for a significant excess is greatly reduced due to the avoidance of an all-sky trial factor. Sensitivity can be further enhanced when information from multiple objects is combined in a so-called 'stacking search'. These methods can be used to test for neutrino excesses correlated to astrophysical populations. One drawback of these methods is that their sensitivity strongly depends on the quality of available multi-wavelength data. Often these catalogues are not complete, particularly in the case of transient or variable objects. 

On the other hand, realtime analysis is a complementary method that inverts this traditional object neutrino relationship. Instead of taking a known object, and asking whether neutrinos are correlated to it, realtime analysis identifies likely-astrophysical neutrinos and seeks to identify coincident astrophysical objects that could potentially have produced the neutrino. The power of realtime searches is that they can, if a possible counterpart is identified, lead to contemporaneous multi-wavelength follow-up that maybe reveal more about a given object. Within this context, the 

However, it should be noted that both realtime and stacking analyses are only sensitive to cases where neutrino sources have detectable EM counterparts. It might however be the case that EM emission from neutrino sources is either absorbed or attenuated, and consequently stacking analyses will be unable to identify such EM-dark neutrino sources. Furthermore, particularly for CCSNe-like populations following the Star Formation Rate (SFR), a large fraction of the astrophysical neutrino might in fact come from unresolved sources.

\subsection{Galactic Neutrino Emission}

As can be seen in Fig, the galactic plane accounts for a significant fraction of EM emission in every wavelength, from radio to high-energy gamma rays. It is therefore natural to suspect that the Milky Way might contribute to the astrophysical neutrino flux. Likely sources of galactic neutrinos are much the same as high-energy gamma rays, namely Supernova Remnants (SNR) and Pulsar Wind Nebulae (PWN). In addition, given that the galaxy should act as a target for extragalactic UHECRs and thus produce secondary neutrinos, it is guaranteed that some galactic contribution should be present in the neutrino flux. Despite this expectation, to date no significant galactic neutrino excess has been found. Given the position of the galactic center in the southern hemisphere, where IceCube muon track datasets are less sensitive, IceCube searches are typically conducted using likely-astrophysical cascades. These searches have in some cases been conducted jointly with the northern-hemisphere ANTARES neutrino observatory, and additionally with HAWC high-energy gamma-ray detector. At this point, limits on the galactic neutrino flux are beginning to constrain reasonable models, above all the standard KRA-$\gamma$ model. Given constraints limiting the galactic contribution to less than n\% of the diffuse flux, we can state with certainty that the astrophysical neutrino flux must be \textbf{predominantly extra-galactic}. 

Fermi Bubbles

\subsection{Blazars}
One long-favoured candidate neutrino source class is Blazars, the subclass of Active Galactic Nucleii (AGN) with relativistic particle jets that point towards the Earth. These blazars have long been known to emit both high-energy  and Very-High Energy (VHE) gamma-rays in the MeV-TeV range. Extensive modeling of these objects, particularly nearby examples such as BL-Lac and Markarian 421 (Mrk 421), have revealed a characteristic SED with two charectiristic "humps", as shown in Figure n. While there is consensus that the lower-energy hump likely arises from synchrotron emission,  the higher-energy one has been explained both by leptonic and hadronic models. Neutrino emission would be expected for the latter in all cases, but never the former.

Since 200n, there have been extensive observations by the Fermi Large Area Telescope (Fermi-LAT), an MeV-GeV gamma-ray satellite telescope. With a significantly increased sensitivity over its predecessors, Fermi has discovered n00 blazars as of 4FGL, adding to the  n from previous mission.

It is now known that blazar emission dominates the high-energy gamma-ray sky. Modelling of the "Extragalactic Background Light" (EBL) typically assumes that 80\% of all gamma-ray emission in the Fermi range is produced by blazars. IACTs have also confirmed that blazars sunch as Mrk 421 are extremely bright at TeV gamma-rays. Although photons at these energies are quickly attenuated during propagation due to interactions with CMB PHOTONS, it is assumed by extrapolation from observations of nearby blazars that more distant ones are also likely TeV-emitters.

 Given the simultaneous production of gamma-rays with neutrinos in hadronic interactions, it is natural to suspect that bright gamma-ray sources, namely blazars, may additionally be neutrino sources. This hypothesis has been tested repeatedly by IceCube, and under the assumption of a linear proportionality, the contribution of all blazars has been constrained to less than 6\% of the astrophysical neutrino flux. This limit additionally accounts for the contribution of unresolved blazars, as well as those in the 3FGL? catalogue that were tested. A more agnostic search on 3FHL blazars constrained their contribution to be less than 20-30\%, without limiting the contribution of unresolved blazars. In both cases, it must be pointed out that this limit is dependent on assumed spectral index. These constraints generally disfavoured blazars as neutrino sources, with fine-tuned models required to generate neutrinos from lower-luminosity unresolved blazars without violating constraints on the brighter resolved blazars. 
 
 However, this interpretation was challenged by the observation of IC170922A, a high-energy neutrino that arrived in spatial and temporal coincidence with a bright gamma-ray flare from blazar TXS 0506+056. A likelihood analysis correlating high-energy neutrinos with the monthly gamma-ray lightcurves of Fermi blazars led to the disfavouring of a chance coincidence at the level of $3 \sigma$. This result implied that, rather than the average gamma-ray flux, high-energy neutrinos might instead be correlated with instantaneous gamma-ray flux. Prompted by this observation, the IceCube collaboration conducted a search for archival neutrino emission from the direction of TXS 0506+056
\subsection{Gamma-Ray Bursts}

LLGRBs+multiplets

\subsection{Tidal Disruption Events (TDEs)}
== Roche Limit==
The tidal radius can best be understood by comparison to the more straightforward concept of the Roche Limit, otherwise known as the Roche Radius. The Roche Radius is defined as the distance at which the self-gravitiational force acting on the outer surface of a body are exactly equal to the gravitational pull of a second body on that same mass element. If a star moves closer than its Roche Radius for an SMBH, there will be a net force on mass elements on the star surface towards the SMBH, and the star will disintegrate.

The Tidal Force of one body acting on another is simply the difference in gravitational force strength between the closer surface of the sphere and its core. The Tidal Force of a black hole of mass M and distance d, acting on a mass element u lying on the closest point of the stellar surface, is thus given by:

$F_{tidal} = \frac{GMu}{(d-r)^{2}} - \frac{GMu}{(d)^{2}}$

$F_{tidal} = \frac{GMu}{d^{2}}((1-\frac{r}{d})^{-2} - 1 )$

Using Binomial Expansion with the reasonable assumption that r<<d, we reach an approximation:

$F_{tidal} \approx \frac{GMu}{d^{2}}\times \frac{2r}{d}$
The Roche Radius for a black hole of mass M for a star of mass m and radius r is thus found by equating this with the self-gravitational force acting on the same mass element. 

$F_{grav} = \frac{Gmu}{r^{2}} = \frac{2GMur}{d^{3}} \approx F_{tidal}$

$ d = r \times (\frac{2M}{m})^{\frac{1}{3}} \approx r_{roche}$

== Tidal Radius ==

This simplistic model does not hold for Tidal Disruption Events. The tidally-disrupted body would need to be in a circular orbit with synchronised spin, but stars tend to have approach Black Holes parabolically. In addition to Angular Momentum, General Relativity should be accounted for. This is particularly relevant for large black holes. A rigorous definition of Tidal radius should account for these things, but in any case, the point at which a star becomes 'tidally disrupted' is difficult to define. An event which only strips a fraction of the outer shell of the star would be ambiguous. Fortunately these definitions differ only by the order of unity from the Roche Radius. In the original paper introducing these calculations by [https://www.nature.com/nature/journal/v333/n6173/abs/333523a0.html Rees (1988)], the tidal radius was defined as the radius at which the mean internal density of the star is exceed by the mean internal density of the orbital volume. In this definition:

$\rho_{star} = \frac{m}{\frac{4}{3} \pi r ^{3}}  = \frac{M}{\frac{4}{3} \pi d ^{3}}= \rho_{volume}$

$ d = r \times (\frac{M}{m})^{\frac{1}{3}} = r_{tidal} $

Thus we see that:

$ r_{tidal} = \frac{r_{roche}}{\sqrt[3]{2}} $

In any definition of a characteristic tidal radius, we nonetheless always find that the radius scales with the cubic root of mass. It is interesting to recall that, in contrast, the Schwarzschild Radius of a black hole scales linearly with its mass. 

$R_{S} = \frac{2GM}{c^{2}}$

Consequently, the Schwarzschild Radius of the Black Hole grows faster as a function of Mass than the tidal radius. There is thus a critical black hole mass, for which the Schwarzschild Radius equals the star's Tidal Radius. Above this, a TDE cannot occur, because the star would have to be wholly swallowed by the black hole in order to be tidally disrupted. Though the exact limit will vary somewhat depending on the mass of the star, using typical values gives us an order-of-magnitude upper limit on TDE-generating black holes of $M < 10^{8} M_{\odot}$.

== Post-Disruption Evolution ==

For TDEs, the accretion of stellar material produces a highly-luminous flare, which is often the cause of discovery. This can be visible in Optical, UV or XRays. The bound mass spirals into the Black hole, but the fall-in time of each mass element in the bound material depends on the Gravitational Potential Energy of the element, leading to a characteristic fall-in rate $ \frac{dM}{dt} \propto t^{\frac{5}{3}}$ relation. This, in turn, can be seen in the light curves of TDEs.

There is evidence to support the existence of jetted TDEs, and these are usually highly-luminous in X rays. However, the classification of candidates into jetted or non-jetted can be unclear.

There is a further proposed model in which an envelope forms around the Black Hole, which could lead to choked jets.

== Neutrino Emission ==

The process for neutrino emission in TDEs, as well as the expected rates and neutrino light curves, are all highly uncertain. A key component of this analysis will be to probe the importance of an accurate time PDF, and the degree to which a minimal-assumption wide box model is the optimal time PDF to use. Even assuming the shape of the expected light curve was well known for each TDE (either the same neutrino light curve for every TDE,  or one closely correlated to the TDE optical or X Ray light curve), there remains a sigificant degree of uncertainty regarding the delay between neutrino emmision and optical emmission. It is possible that the peak neutrino emission is achieved more than 100 days before optical peak, but the expected gap is entirely model-dependent. The following potential models have been proposed for Neutrino Acceleration:

* Jetted TDEs, such as Swift J1644-57
* Choked Jet Scenario

\subsection{Core-Collapse Supernovae (CCSNe)}

\subsection{Superluminous Supernovae}

\subsection{Fast Radio Bursts}

\subsection{Active Galactic Nuclei (AGN) and Starburst Galaxies}

\section{Event Selection}

\subsection{Event Reconstruction}
\subsection{Angular Errors}
\section{Unbinned Likelihood Analysis Formalism}

\section{Results}
\section{Diffuse Astrophysical Neutrino Flux}

\section{Neutrinos from Optical Transients}
\subsection{The Plot}
sensitivity versus depth
Alerts vs ZTF

\section{Realtime Multi-Messenger Analysis}
\subsection{The IceCube Realtime System}
\subsection{Optical Follow-Up of Neutrinos}

\subsection{Optical Follow-Up of Gravitational Waves}

\end{document}