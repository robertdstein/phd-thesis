\setchapterimage[8.5cm]{tde}
\setchapterpreamble[u]{\margintoc}
\chapter{Neutrino Cosmology}
\labch{neutrino_cosmology}
\begin{fquote}[ Kurt Vonnegut][The Design of Experiments][1935]The universe is a big place, perhaps the biggest.
\end{fquote}

Often, the strictest limits on the contribution of a given source class comes not from direct likelihood analysis, but rather from much simpler cosmological arguments. IceCube measures a diffuse astrophysical neutrino flux, setting a limit on the cumulative contribution that can come from a given population. This information can be illustrated by a \emph{Kowalski Plot}, encompassing the product of a local rate and a mean neutrino luminosity per source (see Chapter \ref{ch:sources}). 

As part of this thesis, a software framework was developed to analytically calculate neutrino emission from cosmological populations. This is integrated into the \emph{\href{https://github.com/IceCubeOpenSource/flarestack}{Flarestack}} python package, written by the author \sidecite{flarestack}.

\section{Cosmological Source Rates}

Populations of astrophysical transients are characterised by their \emph{local rate} (how often they occur in the local universe), and their source evolution (how does their rate change as a function of redshift). Rates are typically estimated from unbiased surveys, such as the ZTF Bright Transient Survey. Since many astrophysical transients outlined in Chapter \ref{ch:sources} are ultimately related to stages of stellar evolution, they tend to be strongly correlated to the \emph{Star Formation Rate} (SFR), the rate at which stars are formed at a given redshift.

The SFR can be directly inferred using multi-wavelength surveys of galaxies \sidecite{sfr_madau_14}, exploiting the characteristic spectral ratios to quantify stellar contributions in each galaxy. Alternatively, the SFR rate can be indirectly derived using observed rates of core-collapse supernovae identified in surveys \sidecite{sfr_strolger_15}. The Core-Collapse Supernova (CCSN) Rate is then directly proportional to the SRF, related by a constant of proportionality $k_{cc}$. These two CCSN rate estimates are illustrated in Figure \ref{fig:rhoz}, and broadly agree.
\begin{marginfigure}
	\centering \includegraphics{nu_cosmology/rhoz}
	\caption{Various transient rate densities as a function of redshift.}
	\label{fig:rhoz}
\end{marginfigure}
The SFR peaks at z$\approx$1 because of REASONS, and declines thereafter. It initially has a \emph{positive evolution}, i.e it increases with increasing redshift. 

An alternative driver of evolution is the rate of supermassive black holes (SMBHs), which are connected with many other sources such as blazars, radio galaxies and TDEs. In contrast to the SFR, massive SMBHs are thought to be formed through mergers of smaller SMBHs. The rate of massive SMBHs has a tendency to increase over time, so is larger in the local universe than at higher redshifts. In contrast to the SFR, the SMBH rate this has a \emph{negative source evolution}. Of particular relevance to this thesis is the rate of TDEs, which have been proposed theoretically to follow this negative SMBH rate, as illustrated in Figure \ref{fig:rhoz} \sidecite{Sun:2015bda}. Caution must be taken for this rate, since TDEs are primarily detected in the local universe, so the high-z rates are broadly unconstrained observationally.

With any chosen source rate and neutrino spectrum, we can calculate the corresponding \emph{cumulative neutrino flux} arising from that population \cite{Strotjohann2020Search}. For a transient rate density $\rho (z)$ as a function of redshift, we can calculate the rate of transients per redshift shell. The rate is given per unit time, but the impact of time dilation at higher redshifts suppresses the rate by a factor (1+z). 

\begin{marginfigure}
	\centering \includegraphics{nu_cosmology/rz}
	\caption{Various transient rates as a function of redshift.}
	\label{fig:rz}
\end{marginfigure}

\begin{equation}
R(z) = \rho(z) \times \frac{dV}{dz} \times \frac{1}{1+z}
\end{equation}

These are illustrated in Figure \ref{fig:rz}. The comoving volume of a redshift shell is derived using the \emph{astropy} package. We can then calculate the transient rate in the universe:

\begin{equation}
R_{all} = \int_{0}^{\infty} R(z) dz
\end{equation}

\section{Differential flux and K-corrections}

We can calculate the differential neutrino flux of an object with exactly the same procedure as is used for astronomy \sidecite{hogg_cosmology_99}. The luminosity distance of an object is, by construction, defined such that:

\begin{equation}
D_{L} = \sqrt{\frac{L}{4 \pi S}}
\end{equation}

where L is the integrated luminosity (the \emph{bolometric luminosity}) and S is the integrated energy flux. Thus, for any object, we then find that the integrated energy flux is simply given by:

\begin{equation}
S = \frac{L}{4 \pi D_{L}^{2}}
\label{eq:def_s}
\end{equation}

However, telescopes/detectors are only sensitive in specific energy ranges, so it is often more important to know the flux at a given energy. This quantity, known as the \emph{differential energy flux}, is defined as:

\begin{equation}
S_{E} \equiv \frac{dS(E)}{dE}
\end{equation}

Similarly we define the \emph{differential luminosity} (or \emph{specific luminosity}):

\begin{equation}
L_{E} \equiv \frac{dL(E)}{dE}
\end{equation}

It is here that cosmological effects become important. We often understand well the emission properties $ L_{E'}$ of a source in its local frame. However, the expansion of the universe leads to a process of \emph{redshifting}, so that E', the energy of emitted particles, is not the same as E, the energy at which the particle observed:

\begin{equation}
E' = (1+z) E
\label{eq:redshift}
\end{equation}

For sources with identical intrinsic spectra, $L'(E')$, their corresponding observed spectra will thus depend on the redshift of the object. Expectations for observed spectra need to be adjusted accordingly, in a process known as \emph{k-correction}. By construction, we see from Equation \ref{eq:def_s} that:

\begin{equation}
\int S_{E} dE = \frac{1}{4 \pi D_{L}^{2}} \int L_{E'} dE'
\end{equation}

It then follows that:

\begin{equation}
\frac{dE'}{dE} = (1+z)
\end{equation}

\begin{equation}
 S_{E} dE = \frac{1}{4 \pi D_{L}^{2}}  L_{E'} \frac{dE'}{dE}  dE
\end{equation}

\begin{equation}
S_{E} = \frac{1}{4 \pi D_{L}^{2}} L_{E'} \times (1+z)
\label{eq:kcorrection}
\end{equation}

Equation \ref{eq:kcorrection} is the generic definition of the k-correction, and is valid for both photon and neutrino astronomy. However,  it is often common to consider the special case where source spectra are power laws with some spectral index, $\gamma$:

\begin{equation}
\frac{dN'}{dE'} = \phi_{0} \times E'^{-\gamma}
\label{eq:def_pl}
\end{equation}

\begin{equation}
L_{E'} = \phi_{0} \times E'^{1-\gamma}
\label{eq:def_le}
\end{equation}

We can thus simplify the k-correction even further using Equations \ref{eq:def_le} and \ref{eq:redshift}:

\begin{equation}
L_{E'} = \phi_{0} \times  (E(1+z))^{1-\gamma} = L_{E} \times (1+z)^{1-\gamma}
\end{equation}

\begin{equation}
S_{E} = \frac{1}{4 \pi D_{L}^{2}} \times \phi_{0} \times E^{1 - \gamma} \times (1+z) ^{2-\gamma}
\end{equation}

If we instead wish to consider \emph{differential particle flux}, then by analogy:

\begin{equation}
L_{N'} = \frac{dN'}{dE'} = \phi_{0} \times E'^{-\gamma}
\end{equation}

\begin{equation}
L_{N'} = \phi_{0} \times E'^{-\gamma} \times (1+z)^{-\gamma}
\end{equation}

\begin{equation}
S_{N} = \frac{1}{4 \pi D_{L}^{2}} L_{N} \times (1+z) ^{1-\gamma}
\end{equation}

\begin{equation}
S_{N} = \frac{S_{E}}{E} = \frac{S_{E} (1+z)}{E'} 
\end{equation}

\begin{equation}
S_{N} = \frac{1}{4 \pi D_{L}^{2}} \times \phi_{0} \times E'^{-\gamma}\times (1+z) ^{3-\gamma}
\end{equation}

\begin{marginfigure}
	\centering \includegraphics{nu_cosmology/dnde}
	\caption{Contributed flux at earth as a function of redshift.}
	\label{fig:dnde}
\end{marginfigure}

We first consider the istropic emission of a fixed number of particles, N, by a source. In that case, the particle count per unit area on Earth will ultimately depend only on the comoving distance to that object, $D_{c}$:

\begin{equation}
\frac{dN}{dA} = \frac{N}{4 \pi D_{C}^{2}}
\label{eq:particle_per_area}
\end{equation}

If we instead wish to consider the particle flux, we must differentiate this quantity with respect to time:

\begin{equation}
\frac{dN}{dAdt} = \frac{dN}{dt} \times \frac{1}{4 \pi D_{C}^{2}}
\label{eq:particle_flux}
\end{equation}

At this point, cosmological effects start to become important. In the following, we denote quantites in the source frame with x', while those at Earth are given as x. It is clear that the number of particles itself must be invariant to the frame, i.e N' = N. However, there is in general a process of \emph{cosmic time dilation}, where the passage of time at the source location, t', is slowed when observed at Earth, t:

\begin{equation}
t = (1+z)t'
\label{eq:t}
\end{equation}

\begin{equation}
\frac{dt'}{dt} = \frac{1}{1+z}
\label{eq:dt}
\end{equation}

Often, we only know the emission rate of particles in the source frame, $\frac{dN}{dt'}$, in which case we replace Equation \ref{eq:particle_flux} with:

\begin{equation}
\frac{dN}{dAdt} = \frac{1}{1+z} \times \frac{1}{4 \pi D_{C}^{2}} \times \frac{dN}{dt'}
\end{equation}

It is often more relevant to consider the luminosity of a source, where the intrinsic source luminosity is defined as:

\begin{equation}
L' \equiv \frac{d}{dt'} \left( \int E' dN \right) = \frac{d}{dt'} \left( \int E' \frac{dN}{dE'} dE' \right)
\end{equation}

However, a second cosmological effect must here be accounted for. The expansion of the universe leads to a process of \emph{redshifting}, so that E', the energy of emitted particles, is not the same as E, the energy at which the particle observed:

\begin{equation}
E = \frac{E'}{1+z}
\label{eq:redshift}
\end{equation}

\begin{equation}
\frac{dE}{dE'} = \frac{1}{1+z}
\label{eq:de}
\end{equation}

We will thus observe a redshifted luminosity L as a consequence of the redshifting of each individual particle energy E:

\begin{equation}
L \equiv \frac{d}{dt} \left( \int E dN \right) = \frac{1}{1+z} \times \frac{d}{dt'} \left( \int E dN \right)
\end{equation}

\begin{equation}
L = \frac{1}{1+z} \times \frac{d}{dt'} \left( \int E \frac{dN}{dE'} dE' \right)
\end{equation}

\begin{equation}
L = \frac{1}{(1+z)^{2}} \times \frac{d}{dt'} \left( \int E' \frac{dN}{dE'} dE' \right) = \frac{L'}{(1+z)^{2}}
\end{equation}

Ultimately, the energy flux on Earth, S, will then be given by:

\begin{equation}
S \equiv \frac{dE}{dAdt} =  \frac{1}{(1+z)^{2}} \times \frac{L'}{4 \pi D_{C}^{2}}
\label{eq:energy_flux}
\end{equation}

Given the importance of luminosity and energy flux in astronomy, it is conventional to define a new distance measure known as \emph{luminosity distance}, $D_{L}$:

\begin{equation}
D_{L} \equiv \sqrt\frac{L'}{4 \pi S} = D_{C}(1+z)
\end{equation}

In this case, Equation \ref{eq:energy_flux} is reduced to:

\begin{equation}
S = \frac{L'}{4 \pi D_{L}^{2}}
\end{equation}

However, telescopes/detectors are only sensitive in specific energy ranges, so it is often more important to know the flux at a given energy. This quantity, known as the \emph{differential energy flux}, is defined as:

\begin{equation}
S_{E} \equiv \frac{dS(E)}{dE}
\end{equation}

To calculate the differential flux, we must consider the energy spectrum of the source. The differential energy flux will be given to the product of particle count rate per unit energy at Earth, multiplied the energy-per-particle:

\begin{equation}
S_{E} = E \times \frac{dN(E)}{dEdAdt}
\end{equation}

To conserve particle number despite the impact of redshifting, the observed count rate per unit energy at E, must be equal to the emission rate of particles at energy E':

\begin{equation}
\frac{dN(E)}{dE}= \frac{dN'(E')}{dE}
\end{equation}

Through substitution of Equation \ref{eq:de}:

\begin{equation}
\frac{dN(E)}{dE}= \frac{1}{1+z} \times \frac{dN'(E')}{dE'}
\end{equation}

\begin{equation}
S_{E} = \frac{d}{dAdt} \left( E \frac{dN(E)}{dE} \right)= \frac{1}{1+z} \times \frac{d}{dAdt} \left( E' \frac{dN(E)}{dE} \right)
\end{equation}

\begin{equation}
S_{E} = \frac{1}{(1+z)^{2}} \times \frac{1}{4 \pi D_{C}^{2}} \times \frac{d}{dt} \left( E' \frac{dN'(E')}{dE'} \right)
\end{equation}

\begin{equation}
S_{E} = \frac{1}{(1+z)^{3}} \times \frac{1}{4 \pi D_{C}^{2}} \times \frac{d}{dt'} \left( E' \frac{dN'(E')}{dE'} \right)
\end{equation}

\begin{equation}
S_{E} = \frac{1}{1+z} \times \frac{1}{4 \pi D_{L}^{2}} \times \frac{d}{dt'} \left( E' \frac{dN'(E')}{dE'} \right)
\end{equation}

For convenience, we define the \emph{differential luminosity} (or \emph{specific luminosity}):

\begin{equation}
L'_{E'} \equiv \frac{d}{dt'} \left( E' \frac{dN'(E')}{dE'} \right) = \frac{dL'}{dE'}
\end{equation}

We then see that ultimately:

\begin{equation}
S_{E} = (1+z) \times \frac{L'_{E'}}{4 \pi D_{L}^{2}}
\label{eq:kcorrection}
\end{equation}

The above Equation \ref{eq:kcorrection} is the definition of a \emph{k-correction}, a widely-used formula by which the intrinsic source properties of an object can be converted to the expected differential energy flux. This is a generic definition, and is valid for both photon and neutrino astronomy. It is often common to consider the special case where source spectra are power laws with some spectral index, $\gamma$:

\begin{equation}
\frac{dN'(E')}{dE'dt'} = \phi_{0} \times E'^{-\gamma}
\label{eq:def_pl}
\end{equation}

\begin{equation}
L_{E'} = \phi_{0} \times E'^{1-\gamma}
\label{eq:def_le}
\end{equation}

We can thus simplify the k-correction even further using Equations \ref{eq:def_le} and \ref{eq:redshift}:

\begin{equation}
L'_{E'} = \phi_{0} \times  (E(1+z))^{1-\gamma} = L'_{E} \times (1+z)^{1-\gamma}
\end{equation}

\begin{equation}
S'_{E} = \frac{1}{4 \pi D_{L}^{2}} \times \phi_{0} \times E^{1 - \gamma} \times (1+z) ^{2-\gamma}
\end{equation}

Figure \ref{fig:dnde} shows the contributed diffuse flux at Earth as a function of redshift, assuming each transient contributes $10^{50}$ erg in a power law from 1 GeV to 10 PeV. Ultimately, the diffuse flux measured by IceCube will be the integral of this rate: 

\begin{equation}
\frac{dN}{dE} = \int_{0}^{\infty} \phi(z) dz
\label{eq:nu_flux_tot}
\end{equation}

\begin{marginfigure}
	\centering \includegraphics{nu_cosmology/diffuse_flux}
	\caption{Cumulative flux at earth as a function of redshift.}
	\label{fig:diffuse_flux}
\end{marginfigure}

These cumulative neutrino fluxes are shown in Figure \ref{fig:diffuse_flux}. 

\section{Comparing Source Classes}

The impact of differing the spectral index is illustrated in Figure \ref{fig:CDF_gamma}, using the TDE rate. Given that high-z transients are already suppressed, the overall impact is relatively minor. However, as is clear in both Figure \ref{fig:dnde} and \ref{fig:CDF_rate}, the source evolution heavily impacts the relative contribution of nearby sources to the diffuse flux. A survey complete up to z=0.25 would already account for 60\% of all TDE neutrino emission, whereas it would contain just 20-25\% of CCSN neutrino emission. An alternative interpretation of Figure \ref{fig:CDF_rate} is that the CDF gives the probability to detect a counterpart as a function of redshift for an astrophysical neutrino from each source class.

\begin{marginfigure}
	\centering \includegraphics{nu_cosmology/flux_gamma}
	\caption{CDF as a function of spectral index for TDEs.}
	\label{fig:CDF_gamma}
\end{marginfigure}


\begin{marginfigure}
	\centering \includegraphics{nu_cosmology/diffuse_flux_rates}
	\caption{CDF as a function of source evolution.}
	\label{fig:CDF_rate}
\end{marginfigure}

\begin{marginfigure}
	\centering \includegraphics{nu_cosmology/diffuse_flux_completeness}
	\caption{Completeness correction factor as a function of source evolution.}
	\label{fig:completeness}
\end{marginfigure}

Figure \ref{fig:completeness} illustrates the completeness correction factor as a function of redshift. For a sample complete up to a redshift of z, multiplying the sample flux by the correction factor will yield the total population flux. An alternative interpretation is that this factor is the number of population neutrinos that would need to be followed up before a counterpart was identified, assuming that the follow-up instruments were sensitive up to redshift z. This Figure illustrates most starkly the challenge of optical follow-up of neutrinos, and the relative sensitivity to different source evolutions. For a negative rate, the cumulative neutrino flux of TDEs will be overwhelmingly dominated by nearby sources, while transients that evolve with the Star Formation Rate (e.g CCSNe) have a much larger contribution from more distant sources.

To give a specific example, ZTF is approximately complete in identifying TDEs up to a redshift of 0.15. Per Figure \ref{fig:CDF_rate}, this volume will contain sources responsible for roughly 35\% of the population flux, and from Figure \ref{fig:completeness}, we see that we would expect to find roughly 1/3 of counterparts to TDE neutrinos. In contrast, CCSNe tend to be dimmer, and so we would be complete primarily up to a redshift of z=0.1?. Even neglecting this effect, with SFR-evolution only 10\% of neutrinos will be produced in the volume z<0.15.

To approximately quantify the impact of this effect, we can consider the general power of tests probing this region. The signal, $\mathcal{S}$ is proportional to the fraction of flux belonging to resolvable sources, while the background rate, $\mathcal{B}$, is proportional to the number of sources within this volume.

We consider a neutrino with typical properties for those issued as IceCube realtime alerts. We assume it has a \emph{signalness} of 50\%, i.e that it has a 50\% probability to be of astrophysical origin rather than from atmospheric backgrounds. Such alerts are typically reported with a median angular error of $\sim$1 degree. Any source class will then have a cumulative expectation of 0.5 $\times f_{\textup{tot}}$, where $f_{\textup{tot}}$ is the fraction of the total diffuse astrophysical neutrino flux contributed by that source class. 
		
We consider a number of source classes, listed in Table N. We calculate their expected background rate by considering their sky rate, given as a product of their local rate, integrated to a typical telescope horizon, and multiplied by a typical telescope detection efficiency. For each source, we also multiply the appropriate search window for neutrino emission, yielding a final background density for any point on the sky. Multiplying this by the neutrino localisation area gives us the ultimate background rate per follow-up. We also consider the expected signal per search, given as the neutrino population expectation multiplied by the fraction of flux which is produced by sources accessible to telescopes.  These are illustrated in Figure \ref{fig:nsig_nbkg} for a variety of source classes.

\begin{table*}[]
	\centering
	\begin{tabular}{||c c c c c c c|} 
	\hline
	Source & Max redshift & Length & Search Area & Detection Efficiency & Local Rate & Evolution \\
	& [z] & [yr] & [sq. deg.] & [\%] & [Mpc$^{-3}$ yr$^{-1}$] & \\
	\hline
	TDE (Non-jetted) & 0.15 & 0.10 & 3.14 & 100.00 & 8e-07 & \\
	TDE (Jetted) & 0.50 & 0.50 & 3.14 & 100.00 & 3e-11 & \\
	SN IIP & 0.05 & 0.30 & 3.14 & 100.00 & 5.3e-05 & \\
	SN Ic & 0.05 & 0.03 & 3.14 & 100.00 & 1.8e-05 & \\
	SN IIn & 0.08 & 0.30 & 3.14 & 100.00 & 6.5e-06 & \\
	FBOT & 0.25 & 0.30 & 3.14 & 100.00 & 1e-07 & \\
	GRB & 1.00 & 0.00 & 314.16 & 100.00 & 4.2e-10 & \\
	\hline
	\end{tabular}
	\caption{Summary of assumptions on source classes.}
	\label{tab:source_properties}
\end{table*}{}

\begin{figure}[!ht]
	\centering \includegraphics{nu_cosmology/nsig_nbkg}
	\caption{Nsig vs Nbkg}
	\label{fig:nsig_nbkg}
\end{figure}

The intrinsic signal-to-noise of different source classes can be seen, with the white band indicating the transition from the red signal-dominated regions to the blue background-dominated regions. In general, the probability of observing a coincidence increases from lower left to upper right, while the significance of any coincidence increases from lower right to upper left. For sources with $N_{\textup{bkg}} \lesssim 0.1$, where the probability of multiple background events is negligible, the x axis represents the p-value for a coincidence. By analogy, the y axis represents the probability of observing a coincidence.

Though Figure \ref{fig:nsig_nbkg} contains no analysis of IceCube data, it is particularly notable that the position of sources correspond very directly to the relative constraints that have been placed on those classes by IceCube. Neutrino-specific properties of signalness and localisation will linearly scale all points in the x or y direction respectively, but the relative positions of sources remains unchanged. Those sources near the top of the Figure are those with the strictest existing limits, often at percent-level, while those nearer the lower/lower-right portion are only weakly constrained or could still produce the entire diffuse flux. 

One particularly noteworthy class is FRBs, which are particularly favourable owing to their extremely stringent temporal localisation. Indeed, were telescopes fully efficient at detecting these across the sky, the limits on this class would be perhaps the most stringent of all. However, at the time of the most recent IceCube analysis, 21 FRBs were tested against an estimated rate of 3000 per sky per day, a sample for which no constraint could be placed on the FRB population \sidecite{icrc_frb}. It is precisely because the detection efficiency of FRBs is so low that a tiny fraction of signal is detectable. The expected signal to background ratio for FRBs is very high, and this statement is independent of detection efficiency, so any FRB-neutrino coincidence would be very strong evidence for a physical association. However, because the probability that an FRB counterpart would be detected at all is so low at present, no coincidence is likely to be found. The corresponding FRB (IC) position is illustrated, with an implied detection efficiency of N\%. 

The plot illustrates how well the full unbinned likelihood analysis method used by IceCube for neutrino astronomy can in fact be approximated by a poisson counting experiment.

\begin{figure}[!ht]
	\centering \includegraphics{nu_cosmology/p_value_nfollowup}
	\caption{Statistical significance of excess coincidence as a function of number of follow-up}
	\label{fig:p_value}
\end{figure}
