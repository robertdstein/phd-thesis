\setchapterpreamble[u]{\margintoc}
\chapter{Overflow}
\labch{options}
\section{Introduction}
\section{Neutrino Theory}
\section{The IceCube Neutrino Observatory}
The IceCube Neutrino Observatory is the world's largest neutrino telescope, located at the geographic south pole. It was built as a successor to the AMANDA experiment, which had pioneered the detection of neutrinos in ice. AMANDA served as a proof of concept for the detection of neutrinos, but with a volume of just nkm$^{3}$, it only had sufficient effective area to measure atmospheric neutrinos. 

IceCube, on the other hand, was envisioned as a much larger detector capable of measuring astrophysical neutrinos. It was constructed in phases from 2006 to 2011, eventually reaching a total volume of one cubic kilometer. 
\section{The Zwicky Transient Facility (ZTF)}

\section{Event Selection}

\subsection{Event Reconstruction}
\subsection{Angular Errors}
\section{Unbinned Likelihood Analysis Formalism}

\section{Results}
\section{Diffuse Astrophysical Neutrino Flux}

\section{Neutrinos from Optical Transients}
\subsection{The Plot}
sensitivity versus depth
Alerts vs ZTF

\section{Realtime Multi-Messenger Analysis}
\subsection{The IceCube Realtime System}
\subsection{Optical Follow-Up of Neutrinos}

\subsection{Optical Follow-Up of Gravitational Waves}

\end{document}