\setchapterpreamble[u]{\margintoc}
\chapter{Realtime Multi-Messenger Astronomy}
\labch{options}

In recent years, there has been a significant renewed interest in the study of transient and variable objects in astronomy. Driven primarily by the speed at which objects can evolve and disappear, particularly GRBs and latterly Kilonovae, it is often essential that astronomy can be done with minimal latency. In this vein, it is now commonplace for detectors to automatically issue so-called alerts for observations that meet given criteria, to enable other instruments to rapidly obtain near-simultaneous observations. Realtime alerts are automatically issued by GRB-searching instruments such as Swift-BAT and Fermi-GBM, while gravitaional-wave events are issued by the LIGO-VIRGO observatories, and high-energy neutrino alerts are issued by IceCube as well as ANTARES.

These alerts are all typically issued via  the Gamma-ray Coordination Network (GCN) system, where observatories can subscribe to automatically be notified or point.  An essential component of this process is the additional information published by astronomers, via GCN or Astronomers Telegrams (ATELs), in which follow-up observations are coordinated. There have been two high-profile examples of this, namely the comprehensive followup of GW170817/GRB170817A that led to the first unambiguous observation of a kilonovae, and the followup of high-energy neutrino IC170922A, which led to the identification of TXS 0506+056 as the first candidate source of TeV neutrinos. 

\section{The IceCube Realtime System}
The IceCube Realtime System has been operating since 2016, providing the first source of high-energy neutrino alerts. The first iteration of the alert system consisted of two streams, namely High-Energy Starting Events (HESE) and Extremely High Energy (EHE) events CITE. Each was an established event selection used to identify likely-astrophysical neutrinos cite. Filters to identify relevant events are deployed on computers at the South Pole, and detections are flagged  with low latency. After fast "online" reconstruction algorithms are applied to events, an automated machine-readable "notice" is distributed via the GCN system. In parallel, data from the event is transmitted via satellite to a computing centre in Madison, Wisconsin where a full likelihood scan is performed on the event (see chapter \ref{Event Selection and Reconstruction} for more details). 

The alerts are vetted by humans to asses the event quality, with visually inspection being used to confirm classified topology and event reconstructions. The operating state of the detector is additionally checked. Following these steps, a plain-text GCN circular is distrubuted via the GCN system to confirm the good nature of the alert, and to provide the updated localisation arising from the full scan. 

The first alert issued under this system, HESE alert IC160427A, was found to be in spatial coincidence with an optical transient detected by the Pan-STARRS Observatory while following up the alert. This transient was initially tentaively classified as a Type Ic supernova, for which various models have predicted neutrino emission (see \ref{Neutrino Sources}).  However, the further spectroscopic and photometric evolution indicated that this was  more likely a Type 1a Supernova, for which no neutrino emission would be expected. Nonetheless, dedicated efforts to simulate an ensemble of IC160427A-like events led to the first characterisation of the impact of systematic uncertainties in modelling the polar glacial ice on directional reconstruction with IceCube for high-energy alerts. 

\subsection{Cluster searches}
lalaland

\subsection{OFU and GFU}