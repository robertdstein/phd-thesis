\setchapterpreamble[u]{\margintoc}
\chapter{Overflow}
\labch{options}
\begin{fquote}[Henri Poincare][Lady Windermere's Fan][18??] Without interpolation all science would be impossible. 
\end{fquote}
\begin{fquote}[Feynman][Lady Windermere's Fan][18??] In theory there is no difference between theory and practice. In practise, there is.
\end{fquote}
\begin{fquote}[Douglas Adams][The Hitchiker's Guide to the Galaxy][18??] A neutrino is not a big thing to be hit by. In fact it's hard to think of anything much smaller by which one could reasonably hope to be hit. And it's not as if being hit by neutrinos was in itself a particularly unusual event for something the size of the Earth. Far from it. It would be an unusual nanosecond in which the Earth was not hit by several billion passing neutrinos.
\end{fquote}
\section{The Zwicky Transient Facility (ZTF)}
\begin{fquote}[Oscar Wilde][Lady Windermere's Fan][18??]We are all in the gutter, but some of us are looking at the stars
\end{fquote}

\section{Event Selection}

\subsection{Event Reconstruction}
\subsection{Angular Errors}

ns bias as function of ...

\section{Diffuse Astrophysical Neutrino Flux}

\section{Neutrinos from Optical Transients}
\subsection{The Plot}
sensitivity versus depth
Alerts vs ZTF

Cosmology + The PLOT

Eddington Bias?

As far back as X, when nuclear fusion was proposed as a mechanism to power the sun, it was expected that a flux of solar neutrinos should be detectable on Earth. This prediction was confirmed by early neutrino detectors, ZZ and Z, which measured the flux xyc.
The discovery of a source of neutrinos from beyond the solar system followed soon after, with the detection of nearby supernova SN1987A. Nearby supernova occur either within our galaxy, or in the neighbouring Magellanic clouds, at a rate of a couple per century. In thie case of SN1987A, the detection of the supernova was preceded by simultaneous detections of an elevated neutrino flux in multiple detectors on Earth. The coincident detection of photons and neutrinos marjked the first multi-messenger detection of an astrophysical source.

It is now well-understood that there is a diffuse flux of MeV-neutrinos produced from SN?, and preparations are underway for the inevitable next nearby SN, coordinated by the SuperNova Early Warning System (SNEWS). Given the increased volume of current-generation neutrino detectors, the next nearby supernova will be measured with much greater precision. IceCube itself will contribute to this, through a dedicated supernova detection system. Given the detector geometry, IceCube will not be able to identify individual neutrino events, nor reconstruct their directions. However, an increase in DOM noise will be clearly measurable, and likely with sufficiently resolution to resolve the time-evolution of the signal. 

In both cases, the neutrinos detected at MeV-energies.  However, in light of the discovery of a flux of high-energy astrophysical neutrinos by IceCube, a new branch of astronomy has developed searching for sources of these astrophysical neutrinos. The central fact for neutrino astronomy, at least in the ~TeV-PeV range for which IceCube is sensitive, is that at lower energies there is an overwhelming background, while at higher energies, statistics are poor. This is illustrated in Fig. The power of IceCube to detect an astrophysical flux depends on the degree to which this differs from the atmospheric background. Above all, the expected energy distribution for most astrophysical sources likely differs from the soft-spectrum atmospheric neutrino flux. In addition, the spatial distribution of the background is broadly isotropic, and uniform as a function of right ascension. The temporal distribution of this background, beyond small-scale season variations, is also reasonably isotropic. Foreground fuxes which differ substantially will be most easily detected.

The expected degree of anisotropy in the extragalactic astrophysical neutrino flux strongly depends on the properties of the assumed sources, in particular their density and source evolution. The source evolution describes the rate or density evolution of astrophysical objects as a function of redshift. For sources with a negative source evolution, the density in the local universe is greater than at high redshift, so there are fewer neutrinos produced by distant, unresolved sources. Conversely, for a strongly positive source evolution, we expect a greater fraction of neutrinos to arrive from unresolved sources. When local density and source evolution are coupled, we then find arrive at a flux which anisotropic to varying degrees. Given that the atmospheric backgrounds are broadly isotropic, and uniform as a function of right ascension, the sensitivity of IceCube to a given astrophysical flux will depend strongly on whether that flux is sufficiently spatially anistotropic to distinguish it from atmospheric backgrounds. A higher-density source population, with positive evolution, would be much harder to identify that a low-density one with negative evolution. Anisotropy in neutrino arrival times can be an additional metric to distinguish astrophysical neutrinos. For sources which are variable or transient, neutrino emission is only expected for distinct time periods, eliminating background events.

In general, given knowledge about the background, the most agnostic methods to identify neutrino sources look for clustering within the data without reference to external datasets. Such auto-correlation analyses can be done with either a time-integrated or time-dependent all-sky likelihood scan. The result of such an analysis is a pixelised likelihood map. Comparisons of this map can then be made to expectations from background, either using the single "hottest spot" in the sky, or by comparing the distribution of hotspots. No significant excess was found for either approach, providing significant general constraints on neutrino source populations, . Given the current detector volumes and resolution, as well as the lack of observed lower-significance overfluctuations, it appears that the astrophysical neutrino flux is not sufficiently anisotropic for auto-correlation analyses to discover a neutrino source in the near future.

As an alternative to agnostic searches, specific source hypothesis tests can be substantially more sensitive. Given the position of a known astrophysical object, the threshold for a significant excess is greatly reduced due to the avoidance of an all-sky trial factor. Sensitivity can be further enhanced when information from multiple objects is combined in a so-called 'stacking search'. These methods can be used to test for neutrino excesses correlated to astrophysical populations. One drawback of these methods is that their sensitivity strongly depends on the quality of available multi-wavelength data. Often these catalogues are not complete, particularly in the case of transient or variable objects. 

On the other hand, realtime analysis is a complementary method that inverts this traditional object neutrino relationship. Instead of taking a known object, and asking whether neutrinos are correlated to it, realtime analysis identifies likely-astrophysical neutrinos and seeks to identify coincident astrophysical objects that could potentially have produced the neutrino. The power of realtime searches is that they can, if a possible counterpart is identified, lead to contemporaneous multi-wavelength follow-up that maybe reveal more about a given object. Within this context, the 

However, it should be noted that both realtime and stacking analyses are only sensitive to cases where neutrino sources have detectable EM counterparts. It might however be the case that EM emission from neutrino sources is either absorbed or attenuated, and consequently stacking analyses will be unable to identify such EM-dark neutrino sources. Furthermore, particularly for CCSNe-like populations following the Star Formation Rate (SFR), a large fraction of the astrophysical neutrino might in fact come from unresolved sources.

Neutrino Astronomy with IceCube is rather distinct from more mature branches of astronomy, because we have much less power to distinguish astrophysical signal from background. This regime is to some extent fixed by the raw event rate, in which irreducible atmospheric backgrounds are overwhelmingly dominant over astrophysical neutrinos except at the highest energies. However, it is exacerbated by the limited angular resolution of the detector, where each IceCube event covers a relatively-large area of the sky. Further, owing to the limited effective area of IceCube, there are insufficient numbers of signal-like neutrinos to form cleanly-identifiable clusters. Fundamentally, there is almost no signature in data for which a background origin can be discounted.

It is thus not difficult to find interesting things that are correlated with neutrino arrival directions or times, such as weekends or star signs, because any individual hypothesis will have a small probability to be correlated with  neutrinos by chance. The sum of these many small probabilities can become a large probability, and unless we are careful to correct for this look-elsewhere effect, many spurious correlations will be found. To shield against this, anisotropies in neutrino data are studied through \emph{blind analysis}, in which methods are first developed using simulated datasets. Once an analysis method and hypothesis has been finalised, and the procedure for determining the significance of a result has been fixed, the analysis can then be repeated on real data.